\section{Parallel SGD}
As before let $T$ a set of training samples $\left\lbrace \left( \bm{x}_{i}, \bm{y}_i \right) \right\rbrace_{i=1}^{m} = \left\lbrace z_{i} \right\rbrace_{i=1}^{m}$. Let $C \left( \bm{w} \right)$ be the cost function for our model to learn off
\[
    C \left( \bm{w} \right) = \frac{1}{N} \sum_{i=0}^{m} C_{z_{i}}\left( \bm{w} \right).
\]
For gradient descent algorithms, we wish to find a weight vector $\bm{w}^{\ast}$ that minimizes our cost function, that is
\[
    \bm{w}^{\ast} = \underset{\bm{w} \in \mathbb{R}^{n}}{\arg \min } \sum_{i=0}^{m} C_{z_{i}}\left( \bm{w} \right)
\]
For gradient descent algorithms, we wish to find a weight vector $\bm{w}^{\ast}$ that minimizes our cost function, that is
\[
    \bm{w}^{\ast} = \underset{\bm{w} \in \mathbb{R}^{n}}{\arg \min } \sum_{i=0}^{m} C_{z_{i}}\left( \bm{w} \right)
\]
Let's introduce the notation $G \triangleq \frac{\partial C}{\partial \bm{w}}$ and $G_{z} \triangleq \frac{\partial C_{z}}{\partial \bm{w}}$ to simplify gradient notation as well as $H \triangleq \frac{\partial G}{\partial \bm{w}}$ and $H_{z} \triangleq \frac{\partial G_{z}}{\partial \bm{w}}$ to simplify Hessian notation. At each step of the SGD, a sample $z_{j} = \left( \bm{x}_j , y_{j} \right)$ is uniformly selected from the training set to update the existing weight vector as
\[
    \bm{w}_{t+1} = \bm{w}_{t} - \eta_{t} G_{z_{j}} \left( \bm{w}_{t} \right)
\]
where $\eta_{t}$ is just the learning rate at iteration $t$. Say a process performs SGD of a data set $T_{1}$ to get from a weight $\bm{w}_{g}$ to weight $\bm{w}_{1}$. When processing another training set $T_{2}$, a sequential SGD algorithm would have started at weight $\bm{w}_{1}$ to reach a possibly different weight vector $\bm{w}_{h}$. To parallelize the SDG algorithm we wish to start computing of training set $T_2$ on weight vector $\bm{w}_{1}$ while simultaneously running the training set $T_1$ on weight vector $\bm{w}_{g}$, but $\bm{w}_{1}$ is not know until SGD is finished with $T_1$. So how do we get around this? One method is to soundly combine models from different processes in the hopes of achieving a weight vector had the SGD was instead run sequentially \cite{MalekiSaeed2017PSGD}. This requires adjusting the computation of $T_2$ to account for the staleness $\bm{w}_{1} - \bm{w}_{g}$ in the initial model. To do so, the second model performs its computations instead on $\bm{w}_{g} - \Delta \bm{w}$ where $\Delta \bm{w}$ is an unknown symbolic vector. This allows the second model to run in parallel and not have to wait until $\bm{w}_{1}$ is produced from the succeeding model. Once the first process is done, the second process takes $\Delta \bm{w}$ to be $\bm{w}_{1} - \bm{w}_{g}$. This technique can be easily extended to an arbitrary number of processors.\\[1\baselineskip]
Let $S_{T} \left( \bm{w} \right)$ represent the SGD computation of a training $T$ from an initial weight vector $\bm{w}$, for example $S_{T_{1}} \left( \bm{w}_{g} \right) = \bm{w}_{1}$. To come up with a model combiner we need to think about how we can calculate
\[
    S_{T} \left( \bm{w} + \Delta \bm{w} \right).
\]
Assuming $S_{T}$ is differentiable at $\bm{w} + \Delta \bm{w}$, we get the following by consider the Taylor series of $S_{T}$ about the point $\bm{w} + \Delta \bm{w}$
\[
    S_{T} \left( \bm{w} + \Delta \bm{w} \right) = S_{T} \left( \bm{w} \right) + S_{T} ' \left( \bm{w} \right) \cdot \Delta \bm{w} + \mathcal{O} \left( \left| \Delta \bm{w} \right|^{2} \right).
\]
We will introduce the notation $M_{D} \triangleq S_{T}'$ as the model combiner. In the equation above the model combiner captures the first order information of how a change in $\Delta \bm{w}$ will effect the SGD. When $\Delta \bm{w}$ is sufficiently small, one can neglect higher order terms and only use the model combiner to combine models from different processes. From \cite{MalekiSaeed2017PSGD} we can show that for a sequence of input examples $z_1 , z_2, \ldots , z_{n}$ the model combiner can be computed as
\[
    M_{D} (\bm{w}) = \prod_{i=1}^{n} \left( \bm{I} - \eta_{i} \cdot H_{z_{i}} \left( S_{T_{i-1}} \left( \bm{w} \right) \right) \right)
\]
where $S_{T_{0}} \left( \bm{w} \right) = \bm{w}$. This result can be easily shown by applying the chain rule to $S_{T} \left( \bm{w} \right) = S_{T_{n}} \left( S_{T_{n-1}} \left( \ldots \left( S_{1} \left( \bm{w} \right) \right) \right) \right)$.\\[1\baselineskip]
Thus to create a parallelized SGD, each of the $p$ processors start with the same initial global weight vector $\bm{w}_{g}$ to compute its local model $S_{T_i} \left( \bm{w}_{g} \right)$ and model combiner $M_{T_{i}} \left( \bm{w}_{g} \right)$ in parallel. A subsequent reduction phase computes $\bm{w}_{i}$ by adjusting the input of processor $i$ by adjusting by the staleness introduced in the $i-1$ processor
\[
    \bm{w}_{i} = S_{T_{i}} \left( \bm{w_{g}} \right) + M_{T_{i}} \left( \bm{w}_{g} \right) \cdot \left( \bm{w}_{i-1} - \bm{w}_{g} \right).
\]
The algorithm for parallel SGD is summarised in algorithm \ref{alg:parallel-SGD}

\begin{algorithm}[ht!!!]
    \caption{Parallel SGD}
    \label{alg:parallel-SGD}
    \SetAlgoLined
    \SetKwInOut{Input}{input}\SetKwInOut{Output}{output}
    \Input{Training data $T$, an initial weight vector $\bm{w}_{g}$ and the number of processes to perform the algorithm $p$}
    \Output{Updated weight vector}
    \BlankLine
    $\left\lbrace T_{1} , T_{2}, \ldots , T_{p} \right\rbrace \gets$ an equal partition of $T$\;
    \For{$T_{i} \in \left\lbrace T_{1} , T_{2}, \ldots , T_{p} \right\rbrace$ {\bf concurrently}}{
        Compute $S_{T_{i}} \left( \bm{w_{g}} \right)$ and $M_{T_{i}} \left( \bm{w}_{g} \right)$\;
    }
    \For{$i \in \left\lbrace 1,2, \ldots , p \right\rbrace$}{
        $\bm{w}_{i} \gets S_{T_{i}} \left( \bm{w_{g}} \right) + M_{T_{i}} \left( \bm{w}_{g} \right) \cdot \left( \bm{w}_{i-1} - \bm{w}_{g} \right)$\;
    }
    \KwResult{$\bm{w}_{p}$}
    \BlankLine
\end{algorithm}