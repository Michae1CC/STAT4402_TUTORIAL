\section{Introduction to High Performance Computing}

\begin{itemize}
    \item So all these parallel algorithms seem great and all, but they won't actually benefit us anything if we don't have the computing power to run them!
    \item While commercial bought laptops usually have more than one core in them, their usage is taken up by processes running in the background, so if you ran anyone of these algorithms on your own machine, chances are you won't see much of an improvement in performace
    \item You might be asking now, {\it what sort of computers can run these algorithms to actually see performance improvements?} Well I'm glad you asked! The answer {High Performance Computers}. High Performance Computers (or just HPCs) are very large machines consisting of hundereds or even thousands of cores to run scientific or analytic programs on. The processes on HPCs are monitored by a special operating system so that any jobs that you submit to be run of a HPC gets exactly the number of processes and amount of memory you've asked for (provided that HPC system is able to provide those resources).
    \item As a student taking STAT4402 you should have access to UQ's \texttt{getafix} HPC.
\end{itemize}