% Report content: b813f47d-864f-4422-92c8-e6f81b30efae
% Describe the problem you are solving.  This does not need to be very in depth but should provide the reader with any 
% information they need in order to understand what you are doing.
% You can cite references but we are after *your* statement of the problem not a copy paste from elsewhere.

% Describe your implementation.   This should not be a copy of your source code but should have enough detail to allow the 
% reader to understand how your program works. You should definitely describe any data structures you have employed as well as any 
% optimisations you have attempted.

% Describe how you know that it works.  This applies to both the correctness and performance. "I tested it with a few cases" 
% would be a low quality answer. How many test cases? Why those cases?  If you used optimisations, did they work? (Both in 
% terms of preserving correctness and actually improving performance).

% Description of performance. (Note that this is a bit more flexible than what is listed in the ECP).  What can you say about the performance 
% of your implementation? How does the run time vary with different inputs? You can also include content about 
% profiling or internal timing analysis you have done here.
%%
\documentclass[preprint,12pt]{elsarticle}

%% Use the option review to obtain double line spacing
%% \documentclass[preprint,review,12pt]{elsarticle}

%% Use the options 1p,twocolumn; 3p; 3p,twocolumn; 5p; or 5p,twocolumn
%% for a journal layout:
%% \documentclass[final,1p,times]{elsarticle}
%% \documentclass[final,1p,times,twocolumn]{elsarticle}
%% \documentclass[final,3p,times]{elsarticle}
%% \documentclass[final,3p,times,twocolumn]{elsarticle}
%% \documentclass[final,5p,times]{elsarticle}
%% \documentclass[final,5p,times,twocolumn]{elsarticle}

%% The graphicx package provides the includegraphics command.
\usepackage{graphicx}
%% The amssymb package provides various useful mathematical symbols
% \usepackage[demo]{graphicx}
\usepackage{bm,amssymb,amsmath,amsfonts,mathrsfs,stmaryrd,amssymb,mathtools,subfig,caption,float,minted,longtable,url}
\usepackage[linesnumbered,lined,boxed,vlined,ruled]{algorithm2e}
\usepackage[margin=1.05in]{geometry}
\renewcommand{\baselinestretch}{1.3}
%% The amsthm package provides extended theorem environments
%% \usepackage{amsthm}

%% The lineno packages adds line numbers. Start line numbering with
%% \begin{linenumbers}, end it with \end{linenumbers}. Or switch it on
%% for the whole article with \linenumbers after \end{frontmatter}.
\usepackage{lineno}

\newtheorem{theorem}{Theorem}[section]
\newtheorem{corollary}{Corollary}[theorem]
\newtheorem{lemma}[theorem]{Lemma}

%% natbib.sty is loaded by default. However, natbib options can be
%% provided with \biboptions{...} command. Following options are
%% valid:

%%   round  -  round parentheses are used (default)
%%   square -  square brackets are used   [option]
%%   curly  -  curly braces are used      {option}
%%   angle  -  angle brackets are used    <option>
%%   semicolon  -  multiple citations separated by semi-colon
%%   colon  - same as semicolon, an earlier confusion
%%   comma  -  separated by comma
%%   numbers-  selects numerical citations
%%   super  -  numerical citations as superscripts
%%   sort   -  sorts multiple citations according to order in ref. list
%%   sort&compress   -  like sort, but also compresses numerical citations
%%   compress - compresses without sorting
%%
%% \biboptions{comma,round}

% \biboptions{}

\journal{Journal Name}

\begin{document}

\begin{frontmatter}

%% Title, authors and addresses

\title{STAT4402 Tutorial}

\author{Michael Ciccotosto-Camp}

\address{University of Queensland}

% \begin{abstract}
% Programs which find the Longest Common Substring (LCS) are employed many areas of scientific endeavour; most notably in bio-informatics and computer science. In this report a Dynamic bottom-up algorithm is implemented (in \texttt{main.c}) to find the LCS of two strings in the {\it C} programming language. The most effective optimizations found to decrease run-time included using 1D matrices, conservative subroutine calling, loop-interchanging and loop-unrolling
% \end{abstract}

\end{frontmatter}

\section{Introduction}
Most computational work done within science is impractical to perform on commercial laptops and desktops, typically due to the extremely high memory and processing demands. Hence, almost every university and industry has its own high-performance computer to carry out such strenuous problems. A lot of research within the realm of Machine Learning benefits from access to high performance machinery as most of them require matrix computation (which can be efficiently carried out on GPU clusters) and can be processed in parallel.\\[1\baselineskip]
In this tutorial we will revisit three models covered in the first few weeks of lectures, these being the k-NN classifier, the perceptron and SDG linear regressor. The serial implementations for theses algorithms maybe become fairly inefficient large data sets, so we shall look at some ways in which these two methods can be decomposed and parallelised.

\section{Parallel KNN}

% See ESLII page 33
% See Dirk page
\begin{itemize}
    \item The $k^{th}$ Nearest-Neighbor (k-NN) methods use observations in the training set $T$ closest in feature space to a given unknown sample $\bm{x}$ to directly find its corresponding prediction $\overline{y}$
    \item The prediction for the k-NN classifier is usually calculated as 
    \[
        \overline{y} \left( \bm{x} \right) = \sum_{\bm{x}_{i} \in N_{k} (\bm{x})} y_{i}
    \]
    \item The notion of 'clostest' implies the use of some sort of meteric. More often than not, feature vectors belong to $\mathbb{R}^{n}$ allowing us to use commonly used metrics to define distance between vectors in our feature space. For our purposes, we shall use the Euclidean norm as a measurement of determining how close two feature values are to each other. The Euclidean norm is simply defined as
    \[
        d \left( \bm{x}, \bm{y} \right) = \left( \sum_{i=1}^{n} \left( x_{i} - y_{i} \right)^{2} \right)
    \]
    \item When a unknown sample $\bm{x}$ is to be classified, a k-NN classifier computes the distance between $\bm{x}$ and the other points within the training set $T$. The training data is then sorted by distance and the $k^{th}$ closests training samples are then used to predict $\bm{x}$.
\end{itemize}

\section{Parallel SGD}

\begin{itemize}
    \item As before let $T$ a set of training samples $\left\lbrace \left( \bm{x}_{i}, \bm{y}_i \right) \right\rbrace_{i=1}^{m} = \left\lbrace z_{i} \right\rbrace_{i=1}^{m}$
    \item Let $C \left( \bm{w} \right)$ be the cost function
    \[
        \underset{\bm{w} \in \mathbb{R}^{n}}{\arg \min } \sum_{i=0}^{m} C_{z_{i}}\left( \bm{w} \right)
    \]
\end{itemize}

% SGD theory

% Log onto HPC w/ examples

% Questions

% Answers

% References

%% References without bibTeX database:

\bibliographystyle{model1-num-names}
\bibliography{bib_ref}

\end{document}