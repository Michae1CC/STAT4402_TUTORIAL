% Report content: b813f47d-864f-4422-92c8-e6f81b30efae
% Describe the problem you are solving.  This does not need to be very in depth but should provide the reader with any 
% information they need in order to understand what you are doing.
% You can cite references but we are after *your* statement of the problem not a copy paste from elsewhere.

% Describe your implementation.   This should not be a copy of your source code but should have enough detail to allow the 
% reader to understand how your program works. You should definitely describe any data structures you have employed as well as any 
% optimisations you have attempted.

% Describe how you know that it works.  This applies to both the correctness and performance. "I tested it with a few cases" 
% would be a low quality answer. How many test cases? Why those cases?  If you used optimisations, did they work? (Both in 
% terms of preserving correctness and actually improving performance).

% Description of performance. (Note that this is a bit more flexible than what is listed in the ECP).  What can you say about the performance 
% of your implementation? How does the run time vary with different inputs? You can also include content about 
% profiling or internal timing analysis you have done here.
%%
\documentclass[preprint,12pt]{elsarticle}

%% Use the option review to obtain double line spacing
%% \documentclass[preprint,review,12pt]{elsarticle}

%% Use the options 1p,twocolumn; 3p; 3p,twocolumn; 5p; or 5p,twocolumn
%% for a journal layout:
%% \documentclass[final,1p,times]{elsarticle}
%% \documentclass[final,1p,times,twocolumn]{elsarticle}
%% \documentclass[final,3p,times]{elsarticle}
%% \documentclass[final,3p,times,twocolumn]{elsarticle}
%% \documentclass[final,5p,times]{elsarticle}
%% \documentclass[final,5p,times,twocolumn]{elsarticle}

%% The graphicx package provides the includegraphics command.
\usepackage{graphicx}
%% The amssymb package provides various useful mathematical symbols
% \usepackage[demo]{graphicx}
\usepackage{bm,amssymb,amsmath,amsfonts,mathrsfs,stmaryrd,amssymb,mathtools,subfig,caption,float,minted,longtable,url,hyperref}
\usepackage[linesnumbered,lined,boxed,vlined,ruled]{algorithm2e}
\usepackage[margin=1.05in]{geometry}
\renewcommand{\baselinestretch}{1.3}
%% The amsthm package provides extended theorem environments
%% \usepackage{amsthm}

%% The lineno packages adds line numbers. Start line numbering with
%% \begin{linenumbers}, end it with \end{linenumbers}. Or switch it on
%% for the whole article with \linenumbers after \end{frontmatter}.
\usepackage{lineno}

\newtheorem{theorem}{Theorem}[section]
\newtheorem{corollary}{Corollary}[theorem]
\newtheorem{lemma}[theorem]{Lemma}

\newcommand{\norm}[1]{\left\lVert#1\right\rVert}
%% natbib.sty is loaded by default. However, natbib options can be
%% provided with \biboptions{...} command. Following options are
%% valid:

%%   round  -  round parentheses are used (default)
%%   square -  square brackets are used   [option]
%%   curly  -  curly braces are used      {option}
%%   angle  -  angle brackets are used    <option>
%%   semicolon  -  multiple citations separated by semi-colon
%%   colon  - same as semicolon, an earlier confusion
%%   comma  -  separated by comma
%%   numbers-  selects numerical citations
%%   super  -  numerical citations as superscripts
%%   sort   -  sorts multiple citations according to order in ref. list
%%   sort&compress   -  like sort, but also compresses numerical citations
%%   compress - compresses without sorting
%%
%% \biboptions{comma,round}

% \biboptions{}

\journal{Univerity of Queensland}

\begin{document}

\begin{frontmatter}

%% Title, authors and addresses

\title{STAT4402 Tutorial Paper}

\author{Michael Ciccotosto-Camp}

\address{University of Queensland}

% \begin{abstract}
% {\it I give consent for this to be used as a teaching resource.}
% \end{abstract}

\end{frontmatter}

\begin{center}
    {\it I give consent for this to be used as a teaching resource.}
\end{center}

\section{Introduction}
Most computational work done within science is impractical to perform on commercial laptops and desktops, typically due to the extremely high memory and processing demands. Hence, almost every university and industry has its own high-performance computer to carry out such strenuous problems. A lot of research within the realm of Machine Learning benefits from access to high performance machinery as most of them require matrix computation (which can be efficiently carried out on GPU clusters) and can be processed in parallel.\\[1\baselineskip]
In this tutorial we will revisit three models covered in the first few weeks of lectures, these being the k-NN classifier, the perceptron and SDG linear regressor. The serial implementations for theses algorithms maybe become fairly inefficient large data sets, so we shall look at some ways in which these two methods can be decomposed and parallelised.
\newpage

\section{Parallel KNN}

% See ESLII page 33
% See Dirk page
\begin{itemize}
    \item The $k^{th}$ Nearest-Neighbor (k-NN) methods use observations in the training set $T$ closest in feature space to a given unknown sample $\bm{x}$ to directly find its corresponding prediction $\overline{y}$
    \item The prediction for the k-NN classifier is usually calculated as 
    \[
        \overline{y} \left( \bm{x} \right) = \sum_{\bm{x}_{i} \in N_{k} (\bm{x})} y_{i}
    \]
    \item The notion of 'clostest' implies the use of some sort of meteric. More often than not, feature vectors belong to $\mathbb{R}^{n}$ allowing us to use commonly used metrics to define distance between vectors in our feature space. For our purposes, we shall use the Euclidean norm as a measurement of determining how close two feature values are to each other. The Euclidean norm is simply defined as
    \[
        d \left( \bm{x}, \bm{y} \right) = \left( \sum_{i=1}^{n} \left( x_{i} - y_{i} \right)^{2} \right)
    \]
    \item When a unknown sample $\bm{x}$ is to be classified, a k-NN classifier computes the distance between $\bm{x}$ and the other points within the training set $T$. The training data is then sorted by distance and the $k^{th}$ closests training samples are then used to predict $\bm{x}$.
\end{itemize}
\newpage

\section{Parallel Multiclass Perceptron}

\begin{itemize}
    \item Similar to a definition is class the perception is an algorithm that uses an affine function as a threshold boundary to classify real valued feature vectors
    \item The output of a perceptron is simply given as
    \[
        f(\mathbf{x})=\left\{\begin{array}{ll}
        1 & \text { if } \mathbf{w} \cdot \mathbf{x}+b>0 \\
        0 & \text { otherwise }
        \end{array}\right.
    \]
    where $\bm{w} \in \mathbb{R}^{m}$ with $b$ acting as a bias
    \item A simple generalisation we can make to the perceptron is to allow the perceptron to predict from more than just two classes. This is known as the {\it multiclass perceptron}. A feature representation function maps a feature vector and its correspond label to a real valued vector. This output vector is then dot-producted with  a weight vector $\bm{w}$, similar to the norm perceptron. The prediction value of an unknown feature vector for the multiclass perceptron can be computed as
    \[
        \overline{y} \left( \bm{x} \right) = \underset{y}{\arg \min } \bm{w} \cdot f(\bm{x}, y)
    \]
    \item The iterative procedure to create our weight vector from the training set is also similar, although this time if we make an incorrect prediction we move the weight vector in the direction of $f(\bm{x}, y) - f(\bm{x}, \overline{y})$.
    \item Algorithm for the serial multiclass perceptron is shown in algorithm \ref{alg:serial-perceptron}. Note, for the sake of simplicity a learning rate parameter has been omitted with in the training algorithm
\end{itemize}
\begin{algorithm}[ht!!!]
    \caption{Serial Multiclass Perceptron}
    \label{alg:serial-perceptron}
    \SetAlgoLined
    \SetKwInOut{Input}{input}\SetKwInOut{Output}{output}
    \Input{Training data $T$, an initial weight vector $\bm{w}^{(0)}$}
    \Output{Updated weight vector}
    \BlankLine
    $k \gets 0$;
    
    \While{until convergence}{
    
        \For{$\left( \bm{x}_{i}, y_{i} \right) \in T$}{
        
            $\overline{y} \gets \underset{y}{\arg \min } \bm{w}^{(k)} \cdot f(\bm{x}, y)$\;
            $k \gets k+1$\;
            
            \uIf{$\overline{y} \neq y$}{
                $\bm{w}^{(k+1)} \gets \bm{w}^{(k)} + f(\bm{x}, y) - f(\bm{x}, \overline{y})$\;
            }
        }
    }
    \KwResult{$\bm{w}^{(k)}$}
    \BlankLine
\end{algorithm}
\begin{itemize}
    \item One obvious way to parallize the algorithm from \ref{alg:serial-perceptron} is to split the training set $T$ into $p$ partitions. One can then train an initial weight on each of these partitions independently on the $p$ processes and then take a weighted mixture of the weights from each of the different processes. The mixture coefficents will be given as the vector $\bm{\mu}$ where $\mu_{i} \geq 0$ and $\sum_{i} \mu_{i} = 1$. Algorithm for this naive parallel multiclass perceptron is shown in algorithm \ref{alg:serial-perceptron}.
\end{itemize}
\begin{algorithm}[ht!!!]
    \caption{Naive Parallel Multiclass Perceptron}
    \label{alg:naive-parallel-perceptron}
    \SetAlgoLined
    \SetKwInOut{Input}{input}\SetKwInOut{Output}{output}
    \Input{Training data $T$, an initial weight vector $\bm{w}^{(0)}$}
    \Output{Updated weight vector}
    \BlankLine
    $k \gets 0$\;
    
    $\left\lbrace T_{1} , T_{2}, \ldots , T_{p} \right\rbrace \gets$ an equal partition of $T$\;
    \For{$T_{i} \in \left\lbrace T_{1} , T_{2}, \ldots , T_{p} \right\rbrace$ {\bf concurrently}}{
        $\bm{w}^{(i)} = \operatorname{Serial Multiclass Perceptron} \left( T, \bm{w}^{(0)} \right)$\;
    }
    
    $\bm{w} = \sum_{i} \mu_{i} \bm{w}^{(i)}$\;
    
    \KwResult{$\bm{w}$}
    \BlankLine
\end{algorithm}
\begin{itemize}
    \item While \ref{alg:naive-parallel-perceptron} is easily understood and makes effective use of parallel resources without incurring too much over-head in inter-process communication, there is one large shortcoming of this algorithm. Even for linearly separable datasets $T$, algorithm \ref{alg:naive-parallel-perceptron} does not always converge to a separating hyperplane.
    \item In fact in Question ??? you yourself will be asked to provide a linearly separable dataset for which algorithm \ref{alg:naive-parallel-perceptron} will not converge.
    \item To salvage this parallel algorithm we can make sure each processors is communicating its results before finalization.
    \item That is we can collect weights from processors after each single epoch, mix these the weights together and then redistribute this new weight vector to our processes again for a new epoch.
    \item This new iterative parameter mixing idea is shown in algorithm.
\end{itemize}
\begin{algorithm}[ht!!!]
    \caption{Parallel Multiclass Perceptron}
    \label{alg:parallel-perceptron}
    \SetAlgoLined
    \SetKwInOut{Input}{input}\SetKwInOut{Output}{output}
    \Input{Training data $T$, an initial weight vector $\bm{w}^{(0)}$}
    \Output{Updated weight vector}
    \BlankLine
    $\bm{w} \gets \bm{w}^{(0)}$\;
    \For{$n \in \left\{ 1,2, \ldots , N \right\}$}{
    \For{$T_{i} \in \left\lbrace T_{1} , T_{2}, \ldots , T_{p} \right\rbrace$ {\bf concurrently}}{
            $\bm{w}^{(i,n)} = \operatorname{One Perceptron Epoch} \left( T_{i}, \bm{w} \right)$\;
        }
        $\bm{w} \gets \sum_{i} \mu_{i} \bm{w}^{(i,n)}$\;
    }
    \KwResult{$\bm{w}$}
    \BlankLine
\end{algorithm}
\begin{algorithm}[ht!!!]
    \caption{One Perceptron Epoch}
    \label{alg:one-perceptron-epoch}
    \SetAlgoLined
    \SetKwInOut{Input}{input}\SetKwInOut{Output}{output}
    \Input{Training data $T_{i}$, an initial weight vector $\bm{w}^{(0)}$}
    \Output{Updated weight vector}
    \BlankLine
    $k \gets 0$\;
    $\bm{w} \gets \bm{w}^{(0)}$\;
    \For{$\left( \bm{x}_{i}, y_{i} \right) \in T_{i}$}{
        
        $\overline{y} \gets \underset{y}{\arg \max } \; \bm{w}^{(k)} \cdot f(\bm{x}, y)$\;
        
        \uIf{$\overline{y} \neq y$}{
            $\bm{w}^{(k+1)} \gets \bm{w}^{(k)} + f(\bm{x}, y) - f(\bm{x}, \overline{y})$\;
            $k \gets k+1$\;
        }
    }
    \KwResult{$\bm{w}^{(k)}$}
    \BlankLine
\end{algorithm}
\begin{itemize}
    \item The obvious and main drawback of algorithm \ref{alg:parallel-perceptron} over \ref{alg:naive-parallel-perceptron} this that there is much more inter-process communication overhead in reducing $\bm{w}^{(i,n)}$ and sending new values of $\bm{w}$.
    \item However, we will see that algorithm \ref{alg:parallel-perceptron} will in fact converge for linearly separable training sets.
    \item We can show this in the following Theorem.
\end{itemize}
{\bf Theorem} Assuming a training set $T$ is linearly separable by a margin $\lambda$, let $k_{i,n}$ be the number of mistakes that occur ed on partition $i$ of the training set during the $n^{th}$ epoch of training. For any $N$, when training a perception with fixed parameter mixing in algorithm \ref{alg:parallel-perceptron}
\[
    \sum_{n=1}^{N} \sum_{i=1}^{S} \mu_{i,n} k_{i,n} \leq \frac{R^{2}}{\lambda^2}
\]
Using some notation from before, let $\bm{w}^{(i,n)}$ be the weight acquired on the $n^{th}$ epoch of training by process $i$ and let $\bm{w}^{(i,n)-k}$ be the weight vector for process $i$ $k$ iterations back. Let $\bm{w}^{(avg,n)}$ be the weighted mixture weight for the $n^{th}$ epoch, that is
\[
    \bm{w}^{(avg,n)} = \sum_{i=1}^{S} \mu_{i,n} \bm{w}^{(i,n)}
\]
we can easily enough show that
\begin{equation}
    \label{eqn:R1}
    \bm{u} \cdot \bm{w}^{(i,n)} \leq \bm{u} \cdot \bm{w}^{(avg,n-1)} + k_{i,n} \lambda \tag{R1}
\end{equation}
and
\begin{equation}
    \label{eqn:R2}
    \norm{\bm{w}^{(i,n)}}^2 \leq \norm{\bm{w}^{(avg,n-1)}}^2 + k_{i,n} R^2 \tag{R2}
\end{equation}
Using \ref{eqn:R1} and \ref{eqn:R2} we can inductively show the following results
\begin{equation}
    \label{eqn:H1}
    \bm{u} \cdot \bm{w}^{(i,n)} \geq \sum_{n=1}^{N} \sum_{i=1}^{S} \mu_{(i,n)} k_{i,n} \lambda \tag{H1}
\end{equation}
and
\begin{equation}
    \label{eqn:H2}
    \norm{\bm{w}^{(i,n)}}^2 \leq \sum_{n=1}^{N} \sum_{i=1}^{S} \mu_{i,n} k_{i,n} R^2 \tag{H2}
\end{equation}
we can see that \ref{eqn:R1} implies $\norm{\bm{w}^{(i,n)}} \geq \sum_{n=1}^{N} \sum_{i=1}^{S} \mu_{i,n} k_{i,n} \lambda$ since $\bm{v} \cdot \bm{w} \leq \norm{\bm{u}} \norm{\bm{w}}$ by the Cauchy Schwarz inequality and $\norm{\bm{u}} = 1$. For the base case $\bm{w}^{(avg,1)}$. So using \ref{eqn:R1} and the fact that $\bm{w}^{(avg,0)} = 0$ and
\begin{align*}
    \bm{u} \cdot \bm{w}^{(avg,1)} = \sum_{i=1}^{S} \mu_{i,1} \bm{u} \cdot \bm{w}^{(i,1)} \geq \sum_{i=1}^{S} \mu_{i,1}  k_{i,n} \lambda
\end{align*}
To show \ref{eqn:H2}, the base case can be written as
\begin{align*}
    \norm{\bm{w}^{(avg,1)}}^2 &= \norm{\sum_{i=1}^{S} \mu_{i,1}}^2 \bm{w}^{(i,1)} \\
    &= \sum_{i=1}^{S} \mu_{i,1} \norm{\bm{w}^{(i,1)}}^{2} \\
    &= \sum_{i=1}^{S} \mu_{i,1} k_{i,n} R^2
\end{align*}
This expression is found by first applying Jensen's inequality followed by \ref{eqn:R2} as well as the fact that $\norm{\bm{w}^{(avg,0)}}^2 = 0$. Now considering the general case, $\norm{\bm{w}^{(avg,N)}}^2$
\begin{align*}
    \bm{u} \cdot \bm{w}^{(avg,N)} &= \sum_{i=1}^{S} \mu_{i,N} \left( \bm{u} \cdot \bm{w}^{(i,N)} \right) \\
    &\geq \sum_{i=1}^{S} \mu_{i,N} \left( \bm{u} \cdot \bm{w}^{(i,N-1)} + k_{i,N} \lambda \right) \tag{Using \ref{eqn:H1}} \\
    &= \bm{u} \cdot \bm{w}^{(avg,N-1)} + \sum_{i=1}^{S} \mu_{i,N} \bm{u} \cdot \bm{w}^{(i,N-1)} k_{i,N} \lambda \\
    &= \geq\left[\sum_{n=1}^{N-1} \sum_{i=1}^{S} \mu_{i, n} k_{i, n} \gamma\right]+\sum_{i=1}^{S} \mu_{i, N} k_{i, N} \\
    &= \sum_{n=1}^{N} \sum_{i=1}^{S} \mu_{i, n} k_{i, n} \gamma
\end{align*}
Which thusly proves \ref{eqn:H1}. To show the inductive case for \ref{eqn:H2} we have
\begin{align*}
    \left\|\mathbf{w}^{(\mathrm{avg}, N)}\right\|^{2} &\leq \sum_{i=1}^{S} \mu_{i, N}\left\|\mathbf{w}^{(i, N)}\right\|^{2} \\
    &= \leq \sum_{i=1}^{S} \mu_{i, N}\left(\left\|\mathbf{w}^{(\mathrm{avg}, N-1)}\right\|^{2}+k_{i, N} R^{2}\right) \\
    &= \left\|\mathbf{w}^{(\mathrm{avg}, N-1)}\right\|^{2}+\sum_{i=1}^{S} \mu_{i, N} k_{i, N} R^{2} \\
    &\leq \left[\sum_{n=1}^{N-1} \sum_{i=1}^{S} \mu_{i, n} k_{i, n} R^{2}\right]+\sum_{i=1}^{S} \mu_{i, N} k_{i, N} R^{2} \\
    &= \sum_{n=1}^{N} \sum_{i=1}^{S} \mu_{i, n} k_{i, n} R^{2}
\end{align*}
This shows \ref{eqn:H2}. Combining the results of \ref{eqn:H1} and \ref{eqn:H2} as well as the fact $\left\|\mathbf{w}^{(\operatorname{avg}, N)}\right\|>\mathbf{u} \cdot \mathbf{w}^{(\mathrm{avg}, N)}$
\begin{align*}
    \left[\sum_{n=1}^{N} \sum_{i=1}^{S} \mu_{i, n} k_{i, n}\right]^{2} \gamma^{2} &\leq \left[\sum_{n=1}^{N} \sum_{i=1}^{S} \mu_{i, n} k_{i, n}\right] R^{2} \\
    \sum_{n=1}^{N} \sum_{i=1}^{S} \mu_{i, n} k_{i, n} &\leq \frac{R^2}{\lambda^2}
\end{align*}
This inequality tells us that if the weight are distributed uniformly then the number of mistakes is bounded and that there is enough convergence between iterations to guarantee convergence.
\newpage

\section{Parallel SGD}

\begin{itemize}
    \item As before let $T$ a set of training samples $\left\lbrace \left( \bm{x}_{i}, \bm{y}_i \right) \right\rbrace_{i=1}^{m} = \left\lbrace z_{i} \right\rbrace_{i=1}^{m}$
    \item Let $C \left( \bm{w} \right)$ be the cost function
    \[
        C \left( \bm{w} \right) = \frac{1}{N} \sum_{i=0}^{m} C_{z_{i}}\left( \bm{w} \right)
    \]
    \item For gradient descent algorithms, we wish to find a weight vector $\bm{w}^{\ast}$ that minimizes our cost function, that is
    \[
        \bm{w}^{\ast} = \underset{\bm{w} \in \mathbb{R}^{n}}{\arg \min } \sum_{i=0}^{m} C_{z_{i}}\left( \bm{w} \right)
    \]
    \item We shall also introduct the notation $G \triangleq \frac{\partial C}{\partial \bm{w}}$ and $G_{z} \triangleq \frac{\partial C_{z}}{\partial \bm{w}}$ to simplify gradient notation as well as $H \triangleq \frac{\partial G}{\partial \bm{w}}$ and $H_{z} \triangleq \frac{\partial G_{z}}{\partial \bm{w}}$ to simplify Hessian notation.
    \item At each step of the SGD, a sample $z_{j} = \left( \bm{x}_j , y_{j} \right)$ is uniformly selected from the training set to update the existing weight vector as
    \[
        \bm{w}_{t+1} = \bm{w}_{t} - \eta_{t} G_{z_{j}} \left( \bm{w}_{t} \right)
    \]
    \item where $\eta_{t}$ is just the learning rate at iteration $t$.
    \item Say a process performs SGD of a data set $T_{1}$ to get from a weight $\bm{w}_{g}$ to weight $\bm{w}_{1}$. When processing another training set $T_{2}$, a sequential SGD algorithm would have started at weight $\bm{w}_{1}$ to reach a possibly different weight vector $\bm{w}_{h}$.
    \item To parallelize the SDG algorithm we wish to start computing of training set $T_2$ on weight vector $\bm{w}_{1}$ while simultaneously running the training set $T_1$ on weight vector $\bm{w}_{g}$, but $\bm{w}_{1}$ is not know until SGD is finished with $T_1$. So how do we ge around this?
    \item One method is to soundly combine models from different processes, in the hopes of achieving a weight vector if SGD was simply run sequentially.
    \item This requires adjusting the computation of $T_2$ to account for the staleness $\bm{w}_{1} - \bm{w}_{g}$ in the initial model.
    \item To do so, the second model performs its computations instead on $\bm{w}_{g} - \Delta \bm{w}$ where $\Delta \bm{w}$ is an unknown symbolic vector
    \item This allows the second model to run in parallel and not have to wait until $\bm{w}_{1}$ is produced from the succeeding model.
    \item Once the first process is done, the second process takes $\Delta \bm{w}$ to be $\bm{w}_{1} - \bm{w}_{g}$
    \item This technique can be easily extended to an arbitrary number of processors.
\end{itemize}

\begin{itemize}
    \item Let $S_{T} \left( \bm{w} \right)$ represent the SGD computation of a training $T$ from an initial weight vector $\bm{w}$, for example $S_{T_{1}} \left( \bm{w}_{g} \right) = \bm{w}_{1}$
    \item To come up with a model combiner we need to think about how we can calculate
    \[
        S_{T} \left( \bm{w} + \Delta \bm{w} \right)
    \]
    \item Assuming $S_{T}$ is differentiable at $\bm{w} + \Delta \bm{w}$, we get following by consider the Taylor series of $S_{T}$ about the point $\bm{w} + \Delta \bm{w}$
    \[
        S_{T} \left( \bm{w} + \Delta \bm{w} \right) = S_{T} \left( \bm{w} \right) + S_{T} ' \left( \bm{w} \right) \cdot \Delta \bm{w} + \mathcal{O} \left( \left| \Delta \bm{w} \right|^{2} \right)
    \]
    \item We will introduce the notation $M_{D} \triangleq S_{T}'$ as the model combiner. In the equation above the model combiner captures the first order information of how a change in $\Delta \bm{w}$ will effect the SGD
    \item When $\Delta \bm{w}$ is sufficiently small, one can neglect higher order terms and only use the model combiner to combine models from different processes.
    \item From CITATION we can show that for a sequence of input examples $z_1 , z_2, \ldots , z_{n}$ the model combiner can be computed as
    \[
        M_{D} (\bm{w}) = \prod_{i=1}^{n} \left( \bm{I} - \eta_{i} \cdot H_{z_{i}} \left( S_{T_{i-1}} \left( \bm{w} \right) \right) \right)
    \]
    where $S_{T_{0}} \left( \bm{w} \right) = \bm{w}$. This result can be easily shown by applying the chain rule to $S_{T} \left( \bm{w} \right) = S_{T_{n}} \left( S_{T_{n-1}} \left( \ldots \left( S_{1} \left( \bm{w} \right) \right) \right) \right)$.
\end{itemize}

\begin{itemize}
    \item Thus to create a parallelized SGD, each of the $p$ processors start with the same initial global weight vector $\bm{w}_{g}$ to compute its local model $S_{T_i} \left( \bm{w}_{g} \right)$ and model combiner $M_{T_{i}} \left( \bm{w}_{g} \right)$ in parallel. A subsequent reduction phase computes $\bm{w}_{i}$ by adjusting the input of processor $i$ by adjusting by the staleness introduced in the $i-1$ processor
    \[
        \bm{w}_{i} = S_{T_{i}} \left( \bm{w_{g}} \right) + M_{T_{i}} \left( \bm{w}_{g} \right) \cdot \left( \bm{w}_{i-1} - \bm{w}_{g} \right)
    \]
    \item The algorithm for parallel SGD is summaries below
\end{itemize}

\begin{algorithm}[ht!!!]
    \caption{Parallel SGD}
    \label{alg:parallel-SGD}
    \footnotesize
    \SetAlgoLined
        \SetKwInOut{Input}{input}\SetKwInOut{Output}{output}
        \Input{Training data $T$, an initial weight vector $\bm{w}_{g}$ and the number of processes to perform the algorithm $p$}
        \Output{Updated weight vector}
        \BlankLine
        $\left\lbrace T_{1} , T_{2}, \ldots , T_{p} \right\rbrace \gets$ an equal partition of $T$\;
        \For{$T_{i} \in \left\lbrace T_{1} , T_{2}, \ldots , T_{p} \right\rbrace$ {\bf concurrently}}{
            Compute $S_{T_{i}} \left( \bm{w_{g}} \right)$ and $M_{T_{i}} \left( \bm{w}_{g} \right)$\;
        }
        \For{$i \in \left\lbrace 1,2, \ldots , p \right\rbrace$}{
            $\bm{w}_{i} \gets S_{T_{i}} \left( \bm{w_{g}} \right) + M_{T_{i}} \left( \bm{w}_{g} \right) \cdot \left( \bm{w}_{i-1} - \bm{w}_{g} \right)$\;
        }
        \KwResult{$\bm{w}_{p}$}
        \BlankLine
    \end{algorithm}
\newpage

\section{Introduction to High Performance Computing}

\begin{itemize}
    \item So all these parallel algorithms seem great and all, but they won't actually benefit us anything if we don't have the computing power to run them!
    \item While commercial bought laptops usually have more than one core in them, their usage is taken up by processes running in the background, so if you ran anyone of these algorithms on your own machine, chances are you won't see much of an improvement in performace
    \item You might be asking now, {\it what sort of computers can run these algorithms to actually see performance improvements?} Well I'm glad you asked! The answer {High Performance Computers}. High Performance Computers (or just HPCs) are very large machines consisting of hundereds or even thousands of cores to run scientific or analytic programs on. The processes on HPCs are monitored by a special operating system so that any jobs that you submit to be run of a HPC gets exactly the number of processes and amount of memory you've asked for (provided that HPC system is able to provide those resources).
    \item As a student taking STAT4402 you should have access to UQ's \texttt{getafix} HPC.
\end{itemize}
\newpage

\section{Questions}

% Show that the expected prediction of a KNN is the same as the normal KNN
\renewcommand{\labelenumi}{\textbf{Q\arabic{enumi})}}
\renewcommand{\labelenumii}{\textbf{\alph{enumii})}}
\begin{enumerate}

% KNN questions
\item 
    \begin{enumerate}
        \item Suppose the training set
        \[T = \left\{ \left( \left[ 0,1 \right], 0 \right), \left( \left[ 1,1 \right], 0 \right), \left( \left[ 2,1 \right], 0 \right), \left( \left[ 2,2 \right], 0 \right), \left( \left[ 5,8 \right], 1 \right), \left( \left[ 6,7 \right], 1 \right), \left( \left[ 5,7 \right], 1 \right), \left( \left[ 4,5 \right], 1 \right) \right\}\]
        was partitioned into
        \begin{align*}
            T_{1} &= \left\{ \left( \left[ 2,1 \right], 0 \right), \left( \left[ 2,2 \right], 0 \right), \left( \left[ 5,8 \right], 1 \right), \left( \left[ 6,7 \right], 1 \right) \right\} \\
            T_{2} &= \left\{ \left( \left[ 2,1 \right], 0 \right), \left( \left[ 2,2 \right], 0 \right), \left( \left[ 5,8 \right], 1 \right), \left( \left[ 6,7 \right], 1 \right) \right\}
        \end{align*}
        for a k-NN parallel algorithm using the $2$ closest neighbours and the Manhattan distance metric to classify the an unknown sample $\bm{x} = \left[ 0,0 \right]$ use two processes. What $2$ closest neighbours would be returned by each of the processors?
        
        \item Would the parallel k-NN algorithm still work if only $k-1$ closest neighbours was instead returned by each process?
        
        \item 
        \begin{enumerate}
            \item Implement your own parallel k-NN algorithm from scratch in python, only using the \texttt{threading} and \texttt{queue} libraries to perform parallelism. Use \texttt{sklearn}s \href{https://scikit-learn.org/stable/modules/generated/sklearn.datasets.load_iris.html}{iris dataset} to test your model, you can access this data set using the import \texttt{from sklearn import datasets} and then creating the dataset by using \texttt{datasets.load\_iris()}.
            \item Assuming the completion of the question above, implementing these parallel algorithms yourself can be tedious and error prone. Forturnately many parallel Machine Learning algoirthms and models have been implemented for us already in \texttt{python}s \texttt{sklearn} library. Like \texttt{pytorch}, \texttt{sklearn} is free to use python library that implements many of the Machine Learning tools seen in this course. Use \texttt{sklearn}s \texttt{KNeighborsClassifier} class to train on the \textit{breast cancer} data. Leave $\frac{1}{5}$ of the data aside to provide a test accuracy. Submit this script to \texttt{golith} using a job batch file. Use repeat training with $1-4$ threads and record your results. {\it Hint:} you will need to adjust the \texttt{n\_jobs} parameter in the \texttt{KNeighborsClassifier} to change the number of threads used for running the k-NN algorithm.
        \end{enumerate}
    \end{enumerate}
    
    \item \begin{enumerate}
        \item \begin{enumerate}
            \item Prove Theorem \ref{eqn:R1}
            \item Prove Theorem \ref{eqn:R2}
        \end{enumerate}
        
        \item Give an example data set to show Algorithm \ref{alg:naive-parallel-perceptron} may not converge even when given a linearly separable data
        
        \item Use \texttt{sklearn}s \texttt{Perceptron} class to train on the \textit{digits} data. Leave $\frac{1}{5}$ of the data aside to provide a test accuracy. Submit this script to \texttt{golith} using a job batch file. Use repeat training with $1-4$ threads and record your results. {\it Hint:} you will need to adjust the \texttt{n\_jobs} parameter in the \texttt{Perceptron} to change the number of threads used for running the perceptron algorithm.
    \end{enumerate}
    
    \item \begin{enumerate}
        \item Find the expression for a combiner matrix for a standard linear regression with square loss.
        \item Show that $M_{D} (\bm{w}) = \prod_{i=1}^{n} \left( \bm{I} - \eta_{i} \cdot H_{z_{i}} \left( S_{T_{i-1}} \left( \bm{w} \right) \right) \right)$.
        \item Use \texttt{sklearn}s \texttt{SGDClassifier} class to train on the \textit{digits} data. Leave $\frac{1}{5}$ of the data aside to provide a test accuracy. Submit this script to \texttt{golith} using a job batch file. Use repeat training with $1-4$ threads and record your results. {\it Hint:} you will need to adjust the \texttt{n\_jobs} parameter in the \texttt{SGDClassifier} to change the number of threads used for running the perceptron algorithm.
    \end{enumerate}
\end{enumerate}
\newpage

\section{Answers}

\subsection*{{\bf Q1)}}
\subsubsection*{{\bf a)}}
The neighbours to be returned by the first and second process would be $\left\{ \left( \left[ 2,1 \right], 0 \right), \left( \left[ 2,2 \right], 0 \right) \right\}$ and $\left\{ \left( \left[ 2,1 \right], 0 \right), \left( \left[ 2,2 \right], 0 \right) \right\}$ respectively.

\subsubsection*{{\bf b)}}
No, all $k$ closest neighbours may randomly be given to a single process. If some of these $k$ closest neighbours  were neglected by this process then not all the $k$ closest neighbours will be used in the final reduction step which may give differing results to the serial k-NN algorithm.

\subsubsection*{{\bf c)}}
\subsubsection*{{\bf i)}}
The code for the k-NN algorithm implemented from scratch is shown below
\inputminted[mathescape,
    linenos,
    numbersep=5pt,
    frame=lines,
    framesep=2mm]{python}{src/KNN_parallel.py}

\subsubsection*{{\bf ii)}}
An example batch file for job submission is shown below
\inputminted[mathescape,
    linenos,
    numbersep=5pt,
    frame=lines,
    framesep=2mm]{bash}{src/batch/KNN_1_thread.sh}
    
Code to training a parallel k-NN algorithm using \texttt{sklearn} is given below
\inputminted[mathescape,
    linenos,
    numbersep=5pt,
    frame=lines,
    framesep=2mm]{python}{src/KNN_demo.py}
    
The times for running the parallel algorithm with different numbers of processes is shown below
\begin{table}[h!!!]
\begin{tabular}{l|c|c|c|c}
Number of threads & 1 & 2 & 3 & 4 \\ \cline{2-5} 
Time (sec)        & $0.03754734$ & $0.02993416$ & $0.0252390$ & $0.01833604$
\end{tabular}
\end{table}

\subsection*{{\bf Q2)}}
\subsubsection*{{\bf a)}}
\subsubsection*{{\bf i)}}
Examining line $6$ of Algorithm \ref{alg:one-perceptron-epoch} and knowing that $\bm{u}$ linearly separates the training set by and margin of $\lambda$
\begin{align*}
    \bm{u} \cdot \bm{w}^{(i,n)} &= \bm{u} \cdot \bm{w}^{([i,n] - 1)} + \bm{u} \left( f (\bm{x}_{i}, y_{i}) -  f (\bm{x}_{i}, \overline{y}) \right) \\
    &\geq \bm{w}^{([i,n] - 1)} + \lambda \\
    &\geq \bm{w}^{([i,n] - 2)} + 2 \lambda \\
    &\geq \bm{w}^{(avg, n-1)} + k_{i,n} \lambda
\end{align*}

\subsubsection*{{\bf ii)}}
Again, using line $6$ of Algorithm \ref{alg:one-perceptron-epoch} and using the fact $\bm{w}^{([i,n] - 1)} \left( f (\bm{x}_{i}, y_{i}) -  f (\bm{x}_{i}, \overline{y}) \right) \leq 0$
\begin{align*}
    \norm{\bm{w}^{(i,n)}}^2 &= \norm{\bm{w}^{([i,n] - 1)}}^2 + \norm{f (\bm{x}_{i}, y_{i}) -  f (\bm{x}_{i}, \overline{y})} + 2 \bm{w}^{([i,n] - 1)} \left( f (\bm{x}_{i}, y_{i}) -  f (\bm{x}_{i}, \overline{y}) \right) \\
    &\leq \norm{\bm{w}^{([i,n] - 1)}}^2 + R^2 \\
    &\leq \norm{\bm{w}^{([i,n] - 1)}}^2 + 2R^2 \\
    &\leq \norm{\bm{w}^{([i,n] - 1)}}^2 + k_{i,n} R^2
\end{align*}

\subsubsection*{{\bf b)}}
Consider a binary classification setting where we have two classes $\left\{ 0,1 \right\}$ the training set $T$ has four samples $\left\{ \left( \bm{x}_{1,1}, y_{1,1} \right), \left( \bm{x}_{1,2}, y_{1,2} \right), \left( \bm{x}_{2,1}, y_{2,1} \right), \left( \bm{x}_{2,2}, y_{2,2} \right) \right\}$. Let $y_{1,1} = y_{2,1} = 0$ and $y_{1,2} = y_{2,2} = 1$. Let $\bm{w} \in \mathbb{R}^{6}$ and using block features, define the feature space as
\begin{align*}
    f (\bm{x}_{1,1}, 0) &= \left[ 1 \; 1 \; 0 \; 0 \; 0 \; 0 \right] \quad & f (\bm{x}_{1,1}, 1) &= \left[ 0 \; 0 \; 0 \; 1 \; 1 \; 0 \right] \\
    f (\bm{x}_{1,2}, 0) &= \left[ 0 \; 1 \; 0 \; 0 \; 0 \; 0 \right] \quad & f (\bm{x}_{1,2}, 1) &= \left[ 0 \; 0 \; 0 \; 0 \; 0 \; 1 \right] \\
    f (\bm{x}_{2,1}, 0) &= \left[ 0 \; 1 \; 1 \; 0 \; 0 \; 0 \right] \quad & f (\bm{x}_{2,1}, 1) &= \left[ 0 \; 0 \; 0 \; 0 \; 1 \; 1 \right] \\
    f (\bm{x}_{2,2}, 0) &= \left[ 1 \; 0 \; 0 \; 0 \; 0 \; 0 \right] \quad & f (\bm{x}_{2,2}, 1) &= \left[ 0 \; 0 \; 0 \; 1 \; 0 \; 0 \right]
\end{align*}
Without loss of generality, assume label $1$ is assigned as a tie-breaker we find that $\bm{w}^{1} = \left[ 1 \; 1 \; 0 \; -1 \; -1 \; 0 \right]$ and $\bm{w}^{2} = \left[ 0 \; 1 \; 1 \; 0 \; -1 \; -1 \right]$ for any choice of $\bm{\mu}$ meaning the mixed weight vector will not separate at all points. If $\mu_1$ and $\mu_2$ are both non-zero then all example will be classified as $0$. Furthermore if $\mu_1 = 1$ and $\mu_2 = 0$ then the training sample $(\bm{x}_{2,2}, y_{2,2})$ will be incorrectly classified as $0$. Similarly if $\mu_1 = 0$ and $\mu_2 = 1$ then the training sample $(\bm{x}_{2,2}, y_{2,2})$ will also be incorrectly classified as $0$. However, there is indeed a separating hyperplane for $T$ namely $\bm{w} = \left[ -1 \; 2 \; -1 \; 1 \; -2 \; 1 \right]$.

\subsubsection*{{\bf c)}}
An example batch file for job submission can be found at the public git repo for this tutorial \url{https://github.com/Michae1CC/STAT4402_TUTORIAL} named as \texttt{PERCEPTRON\_1\_thread.sh} under \texttt{src/batch}. Code to training a parallel perceptron algorithm using \texttt{sklearn} can also be found at the public git repo for this tutorial named as \texttt{PERCEPTRON\_demo.py} under \texttt{src}. The times for running the parallel algorithm with different numbers of processes is shown below
\begin{table}[h!!!]
\begin{tabular}{l|c|c|c|c}
Number of threads & 1 & 2 & 3 & 4 \\ \cline{2-5} 
Time (sec)        & $0.23677158$ & $0.25240182$ & $0.216248750$ & $0.15046286$
\end{tabular}
\end{table}

\subsection*{{\bf Q2)}}
\subsubsection*{{\bf a)}}
We know that a single prediction with a single sample for a linear regression has the form
\[
    \overline{y} (\bm{x}) = \bm{x}_{i} \bm{w}
\]
where $\bm{x}_{i}$ include a fixed $1$ in the first position to allow for a bias term. The squared loss for a single prediction is then
\[
    C_{z_{i}} (\bm{w}) = \frac{1}{2} \norm{\bm{x}_{i} \bm{w} - \bm{y}_{i}}^2
\]
We can compute $G_{z_{i}}$ as
\begin{align*}
    G_{z_{i}} (\bm{w}) &= \frac{\partial}{\partial x} C_{z_{i}} (\bm{w}) \\
    &= \frac{\partial}{\partial \bm{w}} \left\langle \bm{x}_{i} \bm{w} - \bm{y}_{i}, \bm{x}_{i} \bm{w} - \bm{y}_{i} \right\rangle \\
    &= \frac{1}{2} \left[ \bm{w}^{\top} \bm{x}_{i}^{\top} \bm{x}_{i} - 2 \bm{y}_{i}^{\top} \bm{x}_{i} \right]
\end{align*}
The hessian $H_{z_{i}}$ may then be computed as
\begin{align*}
    H_{z_{i}} (\bm{w}) &= \frac{\partial}{\partial x} G_{z_{i}} (\bm{w}) \\
    &= \frac{\partial}{\partial \bm{w}} \frac{1}{2} \left[ \bm{w}^{\top} \bm{x}_{i}^{\top} \bm{x}_{i} - 2 \bm{y}_{i}^{\top} \bm{x}_{i} \right] \\
    &= \bm{x}_{i}^{\top} \bm{x}_{i}
\end{align*}
Since $H_{z_{i}} (\bm{w})$ is constant with respect to $\bm{w}$, our combiner matrix simply becomes
\[
    \prod_{i=1}^{n} \left( \bm{I} - \eta_{i} \bm{x}_{i}^{\top} \bm{x}_{i} \right)
\]

\subsubsection*{{\bf b)}}
We know that
\[
    S_{T} \left( \bm{w} \right) = S_{T_{n}} \left( S_{T_{n-1}} \left( \ldots \left( S_{1} \left( \bm{w} \right) \right) \right) \right)
\]
and
\[
    \bm{w}_{t+1} = \bm{w}_{t} - \eta_{t} G_{z_{j}} \left( \bm{w}_{t} \right)
\]
Applying the chain rule to the former expression
\begingroup
\allowdisplaybreaks
\begin{align*}
    M_{T} \left( \bm{w} \right) &= \frac{\partial}{\partial \bm{w}} ( S_{T_{n}} \left( \bm{w} \right) ) \\
    &= \frac{\partial}{\partial \bm{w}} ( S_{z_{n}} ( S_{z_{n-1}} ( \ldots ( S_{z_{2}} (S_{z_{1}} ( \bm{w} )) ))  ) ) \\
    &= \frac{\partial}{\partial \bm{w}} ( S_{z_{n}} ( S_{z_{n-1}} ( \ldots ( S_{z_{2}} ( \bm{w} - \eta_{1} G_{z_{1}} \left( \bm{w} \right) ) ))  ) ) \\
    &= (\bm{I} - \eta_{1} H_{z_{1}} (\bm{w})) \cdot \frac{\partial}{\partial \bm{S}_{T_{1}} (\bm{w})} ( S_{z_{n}} ( S_{z_{n-1}} ( \ldots ( S_{z_{2}} (S_{z_{1}} ( \bm{w} )) ))  ) ) \\
    &= (\bm{I} - \eta_{1} H_{z_{1}} (\bm{w})) \cdot (\bm{I} - \eta_{2} H_{z_{2}} (\bm{S}_{T_{1}} ( \bm{w} ))) \cdot \frac{\partial}{\partial \bm{S}_{T_{2}} (\bm{w})} ( S_{z_{n}} ( S_{z_{n-1}} ( \ldots ( S_{z_{2}} (S_{z_{1}} ( \bm{w} )) ))  ) ) \\
    &\vdots \\
    &= \prod_{i=1}^{n} \left( \bm{I} - \eta_{i} \cdot H_{z_{i}} \left( S_{T_{i-1}} \left( \bm{w} \right) \right) \right)
\end{align*}
\endgroup
as wanted.

\subsubsection*{{\bf c)}}
An example batch file for job submission can be found at the public git repo for this tutorial \url{https://github.com/Michae1CC/STAT4402_TUTORIAL} named as \texttt{SGD\_batch.sh} under \texttt{src/batch}. Code to training a parallel perceptron algorithm using \texttt{sklearn} can also be found at the public git repo for this tutorial named as \texttt{SGD\_demo.py} under \texttt{src}.

The times for running the parallel algorithm with different numbers of processes is shown below
\begin{table}[h!!!]
\begin{tabular}{l|c|c|c|c}
Number of threads & 1 & 2 & 3 & 4 \\ \cline{2-5} 
Time (sec)        & $0.18166565$ & $0.1362492375$ & $0.102186928$ & $0.076640196$
\end{tabular}
\end{table}
\newpage

% Answers

% References

%% References without bibTeX database:

\bibliographystyle{model1-num-names}
\bibliography{bib_ref}

\end{document}